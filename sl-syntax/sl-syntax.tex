% arara: lualatex: { interaction: nonstopmode, synctex: yes }

\documentclass[../logic-text.tex]{subfiles}

\begin{document}


\chapter{Sentential Logic: Syntax}
\label{cha:sent-logic:-synt}

Sentential logic ({\em SL}) is a system of logic that treats statements, or propositions, as fundamental units.
\footnote{Informally, we use `proposition' and `statement' interchangeably.
  Strictly speaking, the proposition is the content, or meaning, that the statement expresses.
  When different sentences in different languages mean the same thing, it is because they express the same proposition.
  Sentential logic is also called \enquote*{propositional logic.}}
\emph{SL} is not just a system of logic, however, it is also a language. There are two very general types of languages, natural and formal. Natural languages are the languages that we ordinarily use to communicate with each other. Examples of natural languages include English, French, German, Mandarin, etc. 

\emph{SL} is a language, and, like any other language, it has a syntax and semantics. 


\end{document}
