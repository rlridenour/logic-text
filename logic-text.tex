% arara: pdflatex: { interaction: nonstopmode, synctex: yes }
% % arara: biber
% % arara: pdflatex: { interaction: nonstopmode, synctex: yes }
% % arara: pdflatex: { interaction: nonstopmode, synctex: yes }


\documentclass[letterpaper,12pt,oneside,onecolumn]{memoir}


\usepackage{graphicx,epstopdf,amsmath,amssymb,url,booktabs,turnstile,multicol,pifont}
\usepackage{prooftrees}
\usepackage{fitch}
\usepackage{microtype}
\usepackage[american]{babel}
\usepackage[sf,sc]{titlesec}
\usepackage{tikz}
\usetikzlibrary{shapes,backgrounds}
\usepackage[autostyle]{csquotes}
\clubpenalty = 10000 % Reduce orphans and widows
\widowpenalty = 10000
\usepackage{enumitem}
\setlist[itemize]{noitemsep} % Comment out for wider separation in lists.
\setlist[enumerate]{noitemsep}

\usepackage[authordate,url=false,isbn=false,backend=biber]{biblatex-chicago} %Change authordate to notes if desired.
\addbibresource{/Users/rlridenour/Dropbox/bibtex/rlr.bib}

\usepackage{todonotes}

% \usepackage{lualatex-math,luatextra}
% \usepackage{libertinus}
% \usepackage{unicode-math}


\usepackage{libertinus-type1}
\usepackage[frenchmath]{libertinust1math}
\usepackage[T1]{fontenc}


\usepackage[unicode=true]{hyperref}

\usepackage{xcolor}
\hypersetup{
    colorlinks,
    linkcolor={red!50!black},
    citecolor={blue!50!black},
    urlcolor={blue!80!black}
}

\newcommand{\lneg}{\neg}
\renewcommand{\land}{\mathbin{\&}}
\newcommand{\lif}{\supset} % For an arrow, use \rightarrow instead of \supset.
\newcommand{\liff}{\equiv} % For a double arrow, use \leftrightarrow instead of \equiv.
\newcommand{\lbr}{[}
\newcommand{\rbr}{]}

% To use T and F in truth tables, uncomment these two lines. For 1 and 0 instead, uncomment the next two.
\newcommand{\true}{\textbf{T}}
\newcommand{\false}{\textbf{F}}

% \newcommand{\true}{1}
% \newcommand{\false}{0}

\usepackage{subfiles}

% This is for producing corner quotes.

\newbox\gnBoxA
\newdimen\gnCornerHgt
\setbox\gnBoxA=\hbox{$\ulcorner$}
\global\gnCornerHgt=\ht\gnBoxA
\newdimen\gnArgHgt
\def\cquote #1{%
\setbox\gnBoxA=\hbox{$#1$}%
\gnArgHgt=\ht\gnBoxA%
\ifnum     \gnArgHgt<\gnCornerHgt \gnArgHgt=0pt%
\else \advance \gnArgHgt by -\gnCornerHgt%
\fi \raise\gnArgHgt\hbox{$\ulcorner$} \box\gnBoxA %
\raise\gnArgHgt\hbox{$\urcorner$}}

\title{Symbolic Logic\\Theory and Applications}
\author{Randy Ridenour}
% \date{}  % Activate to display a given date or no date

\begin{document}


\frontmatter

\begin{titlingpage}
  \thispagestyle{empty}
  % \maketitle
  % \raggedleft % Right align the title page
  
  \rule{1pt}{\textheight} % Vertical line
  \hspace{0.05\textwidth} % Whitespace between the vertical line and title page text
  \parbox[b]{0.75\textwidth}{ % Paragraph box for holding the title page text, adjust the width to move the title page left or right on the page
    
    {\Huge\bfseries Symbolic Logic}\\[2\baselineskip] % Title
    
    {\large\textit{Theory and Applications}}\\[4\baselineskip] % Subtitle or further description
    {\Large\textsc{Randy Ridenour}} % Author name, lower case for consistent small caps
    
    \vspace{0.5\textheight} % Whitespace between the title block and the publisher
  }
\end{titlingpage}



\tableofcontents*

\mainmatter

\subfile{concepts/concepts}
\subfile{categorical-logic/categorical}
\subfile{sl-syntax/sl-syntax}
\subfile{sl-semantics/sl-semantics}
\subfile{sl-trees/sl-trees}
\subfile{sl-derivations/sl-derivations}
\subfile{pl-syntax/pl-syntax}

\printbibliography

\end{document}
