
\chapter{Introduction}
\label{chap:intro}

\section{What is Logic?}
\label{sec:what-logic}

I don't know that there is any accepted definition of logic. Here are four that I've heard:

\begin{enumerate}
\item The study of reasoning
\item The analysis and evaluation of arguments
\item The study of formal languages
\item Study of patterns of truth
\end{enumerate}

\section{Arguments}
\label{sec:arguments}

One fundamental concept in logic is that of the argument.

\begin{description}
\item[Argument] A set of statements, one of which is taken to be supported by the remaining sentences.
\item[Conclusion] The statement in the argument that is being supported.
\item[Premises] The statements in the argument that provide the support for the conclusion.
\end{description}

This might get more complicated later, but for now, think of a statement as a sentence that has a truth-value. Questions and commands, on the other hand, are neither true nor false.

Also, for now, we will assume that there are only two truth values: \textbf{T} and \textbf{F}.

An inference is the process of reasoning from the premises to the conclusion. That is, inference is a psychological process. An argument lists the premises and conclusion.

A distinction needs to be made between deductive arguments and inductive arguments. It's very hard to make this distinction precise, but it's not hard to have an informal understanding of the difference. One can think of the difference in terms of what the argument is intended to establish. If the argument is only intended to establish the probable truth of the conclusion, then it is inductive. If the argument is intended to guarantee the truth of the conclusion, then it is deductive.

Every argument has exactly one conclusion. Very complex arguments may have sub-conclusions, which are themselves inferred from premises. These sub-conclusions then serve as premises for the main conclusion of the argument.

